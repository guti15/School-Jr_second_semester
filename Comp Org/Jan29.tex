% Created 2014-01-29 Wed 11:42
\documentclass[11pt]{article}
\usepackage[utf8]{inputenc}
\usepackage[T1]{fontenc}
\usepackage{fixltx2e}
\usepackage{graphicx}
\usepackage{longtable}
\usepackage{float}
\usepackage{wrapfig}
\usepackage{rotating}
\usepackage[normalem]{ulem}
\usepackage{amsmath}
\usepackage{textcomp}
\usepackage{marvosym}
\usepackage{wasysym}
\usepackage{amssymb}
\usepackage{hyperref}
\tolerance=1000
\author{Robert Gutierrez}
\date{\today}
\title{Jan29}
\hypersetup{
  pdfkeywords={},
  pdfsubject={},
  pdfcreator={Emacs 24.3.1 (Org mode 8.2.5g)}}
\begin{document}

\maketitle
\tableofcontents



Lab 3: 3:30 -5:20 Thursday 2/6

\section{F-E Cycle}
\label{sec-1}
\begin{enumerate}
\item Get Instr
\item Interpret Instrc  :: Binary Instructions
\item Execute Instrc
\item Go to 1
\end{enumerate}

\subsection{Cylces}
\label{sec-1-1}
\begin{itemize}
\item Clock Cycle = CC
\item Clock Period - Time @ start of a CC to start of Next CC
\item Clock Rate (CR) - 1 / Clock Period 
\begin{itemize}
\item \# cycles / seconds
\end{itemize}
\end{itemize}

\begin{center}
\begin{tabular}{ll}
speed & cycle/sec\\
\hline
MHz & 10$^{\text{6}}$\\
GHz & 10$^{\text{9}}$\\
THz & 10$^{\text{12}}$\\
\end{tabular}
\end{center}
\subsection{CPU Time (t) (in seconds) =   ( \# clock cycles ina  program ) / (Clock Rate)}
\label{sec-1-2}


T = (CC) / (CR) 
\begin{itemize}
\item Instruction = I
\item Instruction Count = IC - Depends on Architecture -> Assembly Languge
\item Cylces per Instructions = CPI - Average \# of cylecs  Needed to process each Instruction
\item often Average of ALL Instructions In Instruction set
\end{itemize}


CPI = (CC) / (IC) 

\begin{itemize}
\item CPU Time T = ( IC x CPI) / (CR)
\end{itemize}
example :: 
% Emacs 24.3.1 (Org mode 8.2.5g)
\end{document}
